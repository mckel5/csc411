\documentclass[12pt, letterpaper]{article}

\usepackage{gensymb}

\title{A3: locality \\
    \large Performance estimates \\
    CSC 411 - Noah Daniels}
\author{Marceline Kelly and Nicolas Leffray}

\begin{document}

\maketitle

\section*{Estimated cache hit rate}

\begin{center}
    \begin{tabular}{ l l l }
        \hline
        & row-major access & column-major access \\
        \hline
        90-deg rotation & 2 & 2 \\  
        180-deg rotation & 1 & 3  \\
        \hline
    \end{tabular}
\end{center}
Our \verb|Array2| uses a row-major implementation under the hood.
Since 180 degree rotations map from row to row or column to column, a 180 degree row-major access will have a high cache hit rate because all data is in adjacent memory.
However, this also means that 180 degree column-major accesses will have a low cache hit rate because no data is in adjacent memory.
Both 90 degree rotations have roughly the same, intermediate cache hit rate because they access both adjacent (row-major) and non-adjacent (column-major) memory at some point.

\section*{Operation counts}

\begin{center}
    \begin{tabular}{ l l l l l l l l l }
        \hline
        op & $+-$ & $\times$ & $\div\%$ & comps & loads & hit rate & stores & hit rate \\ 
        \hline
        180R & 5 & 0 & 0 & 1 & 5 & 3/5 & 3 & 3/3 \\
        180C & 5 & 0 & 0 & 1 & 5 & 0/5 & 3 & 0/3 \\
        90R & 3 & 0 & 0 & 1 & 4 & 3/4 & 3 & 3/3 \\
        90C & 3 & 0 & 0 & 1 & 4 & 0/4 & 3 & 0/3 \\
        \hline
    \end{tabular}
\end{center}
These counts are a combination of the operations both within each transform function as well as the iterator created in doing so.
\section*{Expected speed}

\begin{center}
    \begin{tabular}{ l l l }
        \hline
        & row-major & column-major \\
        \hline
        90 degree & 1 & 3 \\  
        180 degree & 2 & 3  \\
        \hline
    \end{tabular}
\end{center}
These estimates are based off of the total loads/stores of each of the above transformations
as well as what \% of them actually hit.
\end{document}